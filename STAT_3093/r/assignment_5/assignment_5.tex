\documentclass[a4paper,11pt]{article}

\usepackage[T1]{fontenc}
\usepackage[utf8]{inputenc}
\usepackage{tabularx}
\usepackage{multirow,array}
\usepackage{amsmath}
\usepackage{amssymb}
\usepackage{mathabx}
\usepackage{tikz}
\usepackage{fancyhdr}
\usepackage{lastpage}
\usepackage[hidelinks]{hyperref}
\usepackage{listings}


\usepackage{color}
\definecolor{lightgray}{rgb}{.9,.9,.9}
\definecolor{darkgray}{rgb}{.4,.4,.4}
\definecolor{purple}{rgb}{0.65, 0.12, 0.82}
\lstdefinelanguage{JavaScript}{
  keywords={const, let, break, case, catch, continue, debugger, default, delete, do, else, false, finally, for, function, if, in, instanceof, new, null, return, switch, this, throw, true, try, typeof, var, void, while, with},
  morecomment=[l]{//},
  morecomment=[s]{/*}{*/},
  morestring=[b]',
  morestring=[b]",
  morestring=[b]`,
  ndkeywords={class, export, boolean, throw, implements, import, this, console, Math, Object},
  keywordstyle=\color{blue}\bfseries,
  ndkeywordstyle=\color{magenta}\bfseries,
  identifierstyle=\color{black},
  commentstyle=\color{purple}\ttfamily,
  stringstyle=\color{teal}\ttfamily,
  sensitive=true
}
\lstset{basicstyle=\footnotesize\ttfamily,breaklines=true}
\lstset{framextopmargin=50pt}

\usepackage{graphicx}
\graphicspath{ {./} }

\usepackage{geometry}
\geometry{left=25mm,right=25mm,bindingoffset=10mm, top=22mm,bottom=30mm, footskip=10mm}
\setlength{\parindent}{0pt}
\setlength{\headsep}{0.5in}


\linespread{1.3}

\pagestyle{fancy}
\fancyhf{}
\rhead{
  STAT 3093 Assignment \#5 \\
  Albert Lockett, 3254354, \href{mailto:k44if@unb.ca}{\texttt{k44if@unb.ca}}
}
\rfoot{Page \thepage / \pageref{LastPage}\\}

\begin{document}
\title{STAT 3093 Assignment \#5}
\author{
  Albert Lockett \\ 
  3254354, 
  \href{mailto:k44if@unb.ca}{\texttt{k44if@unb.ca}}
  }
\date{14/02/2021}

\section*{Problem 1}
\textit{Exercise 46 on page 411.}\newline

The 99\% confidence interval for the standard deviation is:\newline

Lower limit:
\[ 
  \sqrt{\frac{(n - 1)s^2}{\chi^2_{\alpha/2,n-1}}} =
  \sqrt{\frac{(19 -1) 7.234}{38.58}} =
  \textbf{\Large{1.887}}
\]

\vspace{5mm}
Upper limit:
\[
  \sqrt{\frac{(n - 1)s^2}{\chi^2_{1-\alpha/2,n-1}}} =
  \sqrt{\frac{(19 -1) 7.234}{6.844}} =
  \textbf{\Large{4.481}}
\]

\vspace{5mm}
This confidence interval is \textbf{not valid} whatever the nature of the distribution. 
It is only valid for the normal distribution. The formula used to compute the interval relies
on the fact that the distribution of a squred normal distribution is a chi-squared distribution
(from page 315 of textbook).

\vspace{1cm}
\underline{Calculations:}
\vspace{2mm}
\begin{lstlisting}[language=R]
alpha <- 0.01
vals <- c(
  19.75, 21.25, 21.5, 22.50, 23.25, 23.5, 24.00, 24, 24.25, 
  24.5,  25.00, 26.0, 26.25, 26.25, 27.0, 27.75, 28, 28.00, 
  28.25, 30
)
n <- length(vals)
s2 <- var(vals)

chi_high <- qchisq(alpha/2, df=n-1)
chi_low  <- qchisq(1 - alpha/2, df=n-1)

lower <- (n-1)*s2/chi_low
upper <- (n-1)*s2/chi_high

lower_s <- sqrt(lower) # lower limit
upper_s <- sqrt(upper) # upper limit
\end{lstlisting}

\clearpage

\section*{Problem 2}
\textit{Exercise 8 on page 435.}\newline

The null hypothesis $H_0$ is that the average warpage of the special laminate $\mu_s$ will be equal to
the warpage of the regular laminage $\mu_r$.\\
\[ H_0: \hspace{5mm} \mu_s = \mu_r \]

\vspace{8mm}
The alternative hypothesis $H_a$ is that the warpage of the special laminate $\mu_s$ is less than the 
warpage of the regular laminate $\mu_r$.
\[ H_a: \hspace{5mm} \mu_s < \mu_r \]

\vspace{8mm}
A \textbf{type I error} in this context would be determining that the average warpage of the special
laminate is less than the average warpage of the regular laminate when it really is not.\newline

A \textbf {type II error} in this context would be determining that the average warpage of the special
laminate is not less than the warpage of the regular laminate when it really is.

\clearpage

\section*{Problem 3}
\textit{Exercise 12a, b, c on page 436.} 

\subsection*{a)}
The parameter of iterest $\mu$ is the real average breaking distance at 40mph using the new design.\newline

\[ H_0 : \hspace{5mm} \mu = 120 \text{ feet} \]

\[ H_a: \hspace{5mm} \mu < 120 \text{ feet} \]


\subsection*{b)}

The appropriate regection region is \textbf{$R_2$}. 
We want to reject the only if there is a reduction in braking distance so the regection region should 
be lower tailed.

\subsection*{c)}

The signifigance level is:
\[
  \alpha =
  \Phi\left(\frac{\bar{x} - \mu}{\sigma/\sqrt{n}}\right) =
  \Phi\left(\frac{115.20 - 120}{10/\sqrt{36}}\right) = 
  \textbf{\Large{1.9884$\times$10$^{\textbf{-3}}$}}
\]


\vspace{1cm}
To acheive $\alpha = 0.001$:

\[\Phi(z) = 0.001 \implies z = -3.090 \]

\[
z = \frac{\bar{x} - \mu}{\sigma/\sqrt{n}} \implies
\bar{x} = \frac{z\sigma}{\sqrt{n}} + \mu
\]

\[
\bar{x} = \frac{-3.090 \cdot 10}{\sqrt{36}} = \textbf{\Large{114.85}}
\]

\underline{Calculations:}
\begin{lstlisting}[language=R]
  sigma <- 10;
  n <- 36;
  mu <- 120;
  x <- 115.20;
  z <- (115.20 - 120) / (sigma/sqrt(n))
  alpha <-pnorm(z)
  z2 <- qnorm(0.001)
  x2 <- z2 * sigma / sqrt(n) + mu
\end{lstlisting}
\clearpage


\section*{Problem 4}
\textit{Exercise 28 on page 449.} \newline

The data suggests the true average lateral recumebcy time is less than 
20 mins.\newline

Hypotheses
\[ H_0 : \hspace{5mm} \mu = \mu_0 = 20 \text{ mins}\]
\[ H_a : \hspace{5mm} \mu < \mu_0\]

Rejection region
\[ z \le -z_{\alpha}\]
\[ -z_{\alpha} = -z_{0.1} = -1.28\]\newline

Test statistic
\[ 
  z = 
  \frac{\bar{x} - \mu_0}{s/\sqrt{n}} =
  \frac{18.6 - 20}{8.6/\sqrt{73}} =
  -1.39
\]\newline

\Large{The null hypothesis should be rejected}
\[
 -1.39 < -1.28 \implies z < -z_{\alpha}
\]\newline

\large

\underline{Calculations:}
\begin{lstlisting}[language=R]
  n <- 73;
  x_bar <- 18.6;
  S <- 8.6;
  mu_0 <- 20;
  alpha <- 0.1;
  z_alpha <- qnorm(0.1)
  z <- (x_bar - mu_0)/(S/sqrt(n))
  reject_h0 <- z < z_alpha # TRUE
\end{lstlisting}
\clearpage

\section*{Problem 5}
\textit{Exercise 32a on page 449.}

\subsection*{a)}

The data does suggest the population mean differs from 100 using
$\alpha = 0.05$\newline

Hypotheses
\[ H_0 : \hspace{5mm} \mu = \mu_0 \]
\[ H_a : \hspace{5mm} \mu \neq \mu_0 \]\newline

Rejection region
\[ 
  t \ge t_{\alpha/2,n-1} 
  \hspace{5mm}\text{or}\hspace{5mm} 
  t \le -t_{\alpha/2,n-1}
\]
\[ t_{\alpha/2,n-1} = t_{0.05/2,12-1} = 2.201 \]\newline

Test statistic
\[ 
  t = 
  \frac{\bar{x} - \mu_0}{s/\sqrt{n}} =
  \frac{98.46 - 100}{6.142/\sqrt{12}} =
  -0.870
\]\newline

\Large{The null hypothesis should \textbf{not} be rejected}
\[
  -2.201 < -0.870 < 2.201 \implies
  -t_{\alpha/2,n-1} < t < t_{\alpha/2,n-1}
\]\newline
\large

\clearpage
\underline{Calculations:}
\begin{lstlisting}[language=R]
  data <- c(
    105.6,  90.9, 91.2,  96.9,  96.5, 91.3,
    101.1, 105.0, 99.6, 107.7, 103.3, 92.4
  );
  
  # calcuate test statistic value
  n <- length(data);
  x_bar <- mean(data);
  s <- sqrt(var(data));
  mu_0 <- 100;
  t <- (x_bar - mu_0)/(s/sqrt(n))
  
  # calculate rejection region
  alpha <- 0.05;
  t_alpha <- qt(alpha/2, n-1, lower.tail=FALSE)
  
  # check whether should reject
  reject_h0 <- t > t_alpha || t < -1*t_alpha # <- FALSE
\end{lstlisting}

\end{document}











